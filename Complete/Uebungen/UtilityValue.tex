
%\newcommand{\BigO} {\mbox{$\mathcal O$\,}}
%\newcommand{\Eins}{{1\!\!1}}
\renewcommand{\emptyset}{\varnothing}
\renewcommand{\epsilon}{\varepsilon}


\titledquestion{Utility Value} %(vgl. G�nther und Tempelmeier, 2006)

A company is planning to set up a second production facility, for which a pre-selection of four locations has already been made. As a new intern in strategic production planning, you are now tasked with qualitatively evaluating the alternative courses of action within the framework of a Utility Value Analysis.
\begin{enumerate}
	\item Briefly explain the basic idea of the Utility Value Analysis!
	\begin{solution}
		The Utility Value Analysis is a point evaluation procedure, which evaluates all locations with regard to the location features. The advantages of the Utility Value Analysis lie in its simple handling and plausible structure.\\
		The goals here are:
		\begin{enumerate}
			\item the transformation of the qualitative assessment into a uniform quantitative benefit scale
			\item to rank the alternative courses of action
		\end{enumerate}
		
		\uline{Procedure for Utility Value Analysis:}
			\begin{enumerate}
				\item Establish the location features $F_i$ (=criteria).
				\item Determine the weights $w_i$ for each criteria ( typically the sum of the weights is one; i.e. $\sum_i w_i =1$ )
				\item Evaluation of the criteria at each location $X_j$ (=alternatives):
					\begin{itemize}
						\item standardized scale (e.g. points from 1 to 10).
						\item result as a partial utility value $u_{ij}$
					\end{itemize}
				\item Calculation of the ``total Utility'" for each alternative $X_j$ as the weighted sum of the partial utility values $\sum_i w_i u_{ij}$.
				\item	Choose the alternative $X_j$ with highest ``total Utility'"	
			\end{enumerate}
	\end{solution}
\end{enumerate}

The company's internal decision-makers evaluated sites A,B,C,D with regard to six (subjectively) weighted criteria on a standardized scale from 1 (poor) to 9 (very good). The following table summarizes the evaluation results, the so-called partial utility values.
\begin{table}[htbp]
  \centering
    \begin{tabular}{l|cccccc}
    Location Feature & Weighting & A & B & C & D\\
    \hline
(1) Quality/Price of Labor & 0.25 & 9 & 5 & 6 & 8 \\
(2) Connection to Transportation Infrastructure  & 0.20 & 6 & 6 & 5 & 4 \\
(3) Raw Material Reserves & 0.20 & 3 & 4 & 4 & 6 \\
(4) Sales Potential of the Surrounding Area & 0.15 & 7 & 4 & 6 & 5 \\
(5) Distribution of Income/Purchasing Power & 0.10 & 3 & 4 & 6 & 1 \\
(6) Government Subsidies  & 0.10 & 3 & 7 & 7 & 5 \\
    \end{tabular}
\end{table}
\begin{enumerate}
\setcounter{enumi}{1}
	\item What is your recommended course of action for the company according to the Utility Value Analysis? What are the limitations with regard to the meaningfulness of your result?
		\begin{solution}
		
		\uline{Recommended course of action according to the Utility Value Analysis}\\
			\begin{tabular}{c|cccccc}
    Feature & Weighting & A & B & C & D\\
   		 \hline
				(1)& 0.25 & 9 & 5 & 6 & 8 \\
				(2)& 0.20 & 6 & 6 & 5 & 4 \\
				(3)& 0.20 & 3 & 4 & 4 & 6 \\
				(4)& 0.15 & 7 & 4 & 6 & 5 \\
				(5)& 0.10 & 3 & 4 & 6 & 1 \\
				(6)& 0.10 & 3 & 7 & 7 & 5 \\
			\hline
			&1&5.7&4.95&5.5&5.35\\
    	\end{tabular}\\
    	Determine the total Utility Value $\sum_i w_i u_{ij}$\\
    	Location A: $0.25\cdot9+0.2\cdot6+0.2\cdot3+0.15\cdot7+0.1\cdot3+0.1\cdot3=5.7$\\
    	Location B: $0.25\cdot5+0.2\cdot6+0.2\cdot4+0.15\cdot4+0.1\cdot4+0.1\cdot7=4.95$\\
    	Location C: $0.25\cdot6+0.2\cdot5+0.2\cdot4+0.15\cdot6+0.1\cdot6+0.1\cdot7=5.5$\\
    	Location D: $0.25\cdot8+0.2\cdot4+0.2\cdot6+0.15\cdot5+0.1\cdot1+0.1\cdot5=5.35$\\
    	Choose Location A, as $5.7=max\left\{5.7;~4.95;~5.5;~5.35\right\}$.\\
    	
		\uline{Criticism of Utility Value Analysis:}	\\
		\begin{itemize}
			\item Interpretation of total utility: total utility is not meaningfully interpretable in economic terms.
			\item Interchangeability of different objectives: Poor evaluation in one criterion can be fully compensated by other criteria (e.g. location D regarding labor market and purchasing power)
			\item Independence of criteria (e.g. 4 and 5)
			\item Objective separation can possibly distort the criteria weighting
			\item Weights do not mean importance per se, but are to be chosen depending on the scale
			\end{itemize}
			\end{solution}
\end{enumerate}