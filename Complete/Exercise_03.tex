%%%%%%%%%%%%%%%%%%%%%%%%%%%%%%%%%%%%%%%%%%%%%%%%%%%%%%
%% Klausurmakro                 Version 07.09.2010  %%
%%%%%%%%%%%%%%%%%%%%%%%%%%%%%%%%%%%%%%%%%%%%%%%%%%%%%%
%% RWTH Aachen                                      %%
%% Lehrstuhl f�r medizinische Informationstechnik   %%
%% Thomas Schlebusch schlebusch@hia.rwth-aachen.de  %%
%%                                                  %%
%% Universit�t Mainz                                %%
%% Lehrstuhl f�r BWL, insb. Logistikmanagement      %%
%% Claudia Bode claudia.bode@uni-mainz.de           %%
%%%%%%%%%%%%%%%%%%%%%%%%%%%%%%%%%%%%%%%%%%%%%%%%%%%%%%
%% English Translation: Jasper Hatton  [26.09.2023] %%
%%%%%%%%%%%%%%%%%%%%%%%%%%%%%%%%%%%%%%%%%%%%%%%%%%%%%%


%Mit Musterl�sung
\documentclass[answers]{exam}
%Ohne Musterl�sung
%\documentclass{exam}

\newboolean{showgap}
%\setboolean{showgap}{true}
\setboolean{showgap}{false}


\input header


\begin{center}
{\Large Exercise 03}
\end{center}
\footnotesize

\input prepictex
\input pictexwd
\input postpictex


\newcommand{\BigO} {\mbox{$\mathcal O$\,}}
\newcommand{\Eins}{{1\!\!1}}
\renewcommand{\emptyset}{\varnothing}
\renewcommand{\epsilon}{\varepsilon}


\begin{questions}

\input Uebungen/1Median_Manhattan_Ve

\input Uebungen/1Median_Euclidean

\input Uebungen/Miehle_Method



%%\ifprintanswers
%%\newpage
%%\fi
%\titledquestion{Transportation-Location-Problem}
%Ein Anbieter von luxusorientierten Konsumg�tern beliefert vier europ�ische Handelsh�user �ber ein Zentrallager, wobei sich der �berwiegende Teil der Nachfragemenge traditionell auf die beiden westeurop�ischen Abnehmer $i=1,2$ mit den Koordinaten $(a_i,b_i)=(3,0),(3,4)$ konzentriert. Seit einiger Zeit zeichnet sich jedoch eine mengenm��ige Ann�herung der mittel- und osteurop�ischen Kunden $i=3,4$ mit den Koordinaten $(a_i,b_i)=(5,0),(5,4)$ an westeurop�ische Verh�ltnisse ab. Out-of-Stock-Situationen beugt das Unternehmen seit jeher mit einer bedarfs\-orientierten Anpassung der Produktions- und Lagerkapazit�ten vor. In Zukunft wird aus Sicht der Gesch�ftsf�hrung ein Problem darin gesehen, das prognostizierte, j�hrliche Transportaufkommen $w_i=100,200,225,75$ so zu bew�ltigen, dass die damit verbundenen Transportaktivit�ten insgesamt minimal bleiben. Man erw�gt daher eine Restrukturierung des Distributionssystems, einschlie�lich aller bisherigen Lieferbeziehungen. Insbesondere sollen zwei auf der "`gr�nen Wiese"' zu planende dezentrale Regionallager das alte Zentrallager ersetzen. Sie als Experte f�r quantitative Planung sollen die Gesch�ftsf�hrung bei der Wahl der neuen Lagerstandorte unterst�tzen. Gehen Sie dabei von der Luftlinienentfernung als geeignetem Abstandsma� aus.\\
%
%\begin{enumerate}	
%	\item Formulieren Sie ein Modell f�r das Transportation-Location-Problem.
%	\begin{solution}
%	\uline{Gegeben:}
%		\begin{itemize}
%			\item Indizes: $i\in \left\{1,2,3,4\right\}$
%			\item Kundenorte: $\left(3,0\right)$, $\left(3,4\right)$, $\left(5,0\right)$, $\left(5,4\right)$
%			\item Transportmengen: $w_1=100$, $w_2=200$, $w_3=225$, $w_4=75$
%			\item Anzahl der Zentrallager: $p=2$
%			\item Distanzfunktion: Euklidisch
%		\end{itemize}
%	\uline{Gesucht:}
%		\begin{itemize}
%			\item Koordinaten $(x_j,y_j)$ der Zentrallager (\textbf{Lokation})
%			\item Zuordnungen $w_{ij}$, welcher Teil des Bedarfs $w_i$ von Kunde $i$ durch Standort $j$ abgedeckt wird (\textbf{Allokation})
%		\end{itemize}
%	\uline{Modell:}\\
%	\begin{align*}
%	\min&\sum_{j=1}^p\sum_{i=1}^m w_{ij}\sqrt{\left(x_j-a_i\right)^2+\left(y_j-b_i\right)^2}\\
%	\text{s.t. }&\sum_{j=1}^p w_{ij}=w_i \quad \forall i \\
%	&w_{iv}\geq 0 \quad \forall i,j\\
%	&x_j,y_j \in\R \quad \forall j
%	\end{align*}
%	
%	Beachte: Die Zielfunktion ist nicht-linear (Multiplikation von Variablen)
%	\end{solution}
%	
%%	\ifprintanswers
%%	\newpage
%%	\fi
%	\item F�hren Sie eine Iteration der Cooper-Heuristik f�r den unkapazitierten Fall aus, beginnend mit den Startkoordinaten $(x^1_1,y^1_1,x^1_2,y^1_2)=(4,0,4,4)$, und l�sen Sie das Steiner-Weber-Problem in $\R^1$.
%	\begin{solution}
%	
%		\uline{Vorgehensweise der Cooper-Heuristik:}	
%		\begin{itemize}
%			\item W�hle $p$ Standorte als Startl�sung
%			\item (Re-)Allokation f�r gegebene Standorte (kapazitiert: Transportproblem l�sen, unkapazitiert: einfache Inspektion)
%			\item Relokation bei gegebener Allokation (L�se $p$ Steiner-Weber-Probleme)
%			\item Iteriere bis sich die Standorte nicht mehr ver�ndern
%		\end{itemize}
%		
%		\uline{1. Iteration}\\
%		Gegeben Startkoordinaten $\left(x_1^1,y_1^1\right)=\left(4,0\right)$, $\left(x_2^1,y_2^1\right)=\left(4,4\right)$ und Kundendaten				
%\begin{center}
%		\begin{tabular}{c|cccc}
%		i&1&2&3&4\\
%		\hline
%		$a_i$&3&3&5&5\\
%		$b_i$&0&4&0&4\\
%		$w_i$&100&200&225&75\\
%		\end{tabular}	
%\end{center}
%		
%		\uline{Reallokation}: zun�chst Berechnung der Luftliniendistanzen	
%\begin{center}
%	\begin{tabular}{c|cccc}
%		$c_{ij}:=d^2$&i=1&i=2&i=3&i=4\\
%		\hline
%		j=1&1&4,123&1&4,123\\
%		j=2&4,123&1&4,123&1
%		\end{tabular}
%\end{center}
%		
%		Dann optimale Anbindung der Kunden durch einfache Inspektion:
%\begin{center}
%		\begin{tabular}{c|cccc}
%		$w_{ij}$&i=1&i=2&i=3&i=4\\
%		\hline
%		j=1&100&0&225&0\\
%		j=2&0&200&0&75\\
%		\end{tabular}\\
%\end{center}
%		
%		\uline{Relokation}:
%		\begin{itemize}
%			\item Standort $j=1$ sind folgende Kunden zugeordnet:
%					\begin{center}
%					\begin{tabular}{c|cc}
%					i&1&3\\
%					\hline
%					$a_i$&3&5\\
%					$b_i$&0&0 (!)\\
%					\end{tabular}
%					\end{center}
%			\item L�se eindimensionales Weber-Problem f�r Standort $j=1$ mit Manhattan-Distanz:
%	
%			Berechne $W/2=\left(100+225\right)/2=162,5$\\
%			
%			\begin{tabular}{c|c|c|c}
%			i&$a_i$&$w_i$&KumBedarf\\
%			\hline
%			1&3&100&100\\
%			3&5&225&$325 > 162,5$ (*)\\
%			\end{tabular}\\
%			
%			$x^2_1=5$, $y^2_1=0$
%			
%			\item Standort $j=2$ sind folgende Kunden zugeordnet:
%\begin{center}
%	\begin{tabular}{c|cc}
%	i&2&4\\
%	\hline
%	$a_i$&3&5\\
%	$b_i$&4&4 (!)\\
%	\end{tabular}
%\end{center}
%
%			\item L�se eindimensionales Weber-Problem f�r Standort $j=2$ mit Manhattan-Distanz:
%						
%			Berechne $W/2=\left(200+75\right)/2=137,5$\\
%			
%				\begin{tabular}{c|c|c|c}
%				i&$a_i$&$w_i$&KumBedarf\\
%				\hline
%				2&3&200& $200 > 137,5$ (*)\\
%				4&5&75&275\\
%				\end{tabular}
%		
%			$x^2_2=3$, $y^2_2=4$
%				
%		\end{itemize}
%	\end{solution}
%\end{enumerate}

\end{questions}
\end{document}
